\documentclass[12pt]{article}
\usepackage[romanian]{babel}
\usepackage[utf8x]{inputenc}
\usepackage{indentfirst}
\usepackage{amsthm}
\usepackage{amsfonts}
\usepackage{amssymb}
\usepackage{tikz}
\usepackage{geometry}
\usepackage{mathtools}
\usepackage{chngcntr}
\usepackage{caption}
\usepackage{subcaption}
\usepackage{setspace}
\usepackage{enumitem}

\geometry{
	a4paper,
	lmargin=3cm,
	rmargin=2cm,
	tmargin=3cm,
	bmargin=2cm
}
\usetikzlibrary{arrows, shapes, automata, petri, positioning}

\theoremstyle{definition}
\newtheorem{theorem}{Teoremă}
\counterwithin*{theorem}{subsection}
\newtheorem{definition}{Definiție}
\counterwithin*{definition}{subsection}
\newtheorem{remark}{Propoziție}
\counterwithin*{remark}{subsection}
\newtheorem{lema}{Lema}
\newtheorem{algo}{Algoritmul}
\counterwithin*{algo}{subsection}


\counterwithin*{lema}{subsection}
\linespread{1.3}

\begin{document}
	\begin{titlepage}
		\begin{center}
			\vspace{1cm}
			``ALEXANDRU IOAN CUZA" UNIVERSITY OF IAȘI
			\
			\\
			\begin{large}
				\textbf{FACULTY OF COMPUTER SCIENCE}\\
			\end{large}		
			\vspace{2.5cm}
			\includegraphics{fii.png}
			\
			\\
			\vspace{1cm}
			MASTER'S DEGREE
			\begin{large}
				\
				\\
				\vspace{1.5cm}
				\textbf{Human Gait Recognition}
			\end{large}
			\\
			\vspace{1.5cm}
			\textbf{proposed by}
			\\
			\vspace{1.5cm}
			\textit{\textbf{Rareș-Alexandru Stan}}
			\\
			\vspace{2cm}
			\textbf{Session:} \textit{July, 2019}
			\\
			\vspace{1.5cm}
			\textbf{Scientific Coordinator}
			\
			\\
			\
			\\
			\textbf{Lect. Dr. Ignat Anca}
			
		\end{center}
	\end{titlepage}
	\newpage
	
	\begin{titlepage}
		\begin{center}
			\textbf{``ALEXANDRU IOAN CUZA" UNIVERSITY OF IAȘI}
			\
			\\
			\textbf{FACULTY OF COMPUTER SCIENCE}\\
			\vspace{6cm}
			\
			\\
			\begin{huge}
				\textbf{Human Gait Recognition}
			\end{huge}
			\\
			\begin{large}
				\vspace{3cm}
				\textit{\textbf{Rareș-Alexandru Stan}}
				\\
				\vspace{3cm}
				\textbf{Session:} \textit{July, 2019}
				\\
			\end{large}
			\vspace{3cm}
			\textbf{Scientific Coordinator}
			\
			\\
			\textit{\textbf{Lect. Dr. Ignat Anca}	}
		\end{center}
	\end{titlepage}
	\newpage
	\tableofcontents
	\newpage
	
	\section{Introduction}
	\addcontentsline{toc}{section}{Introduction}
	\vspace{1cm}
	
	Gait is the movement pattern of the limbs during walking over a solid surface. It varies based on speed, terrain, maneuvering or efficiency of energy. This movement is unique for each human and can be used for recognizing persons from afar, without the need of their cooperation or physical contact, whereas fingerprint, iris or facial do need the physical access or their cooperation \cite{biometrics-comparison}.
	
	There are three main categories in which recognition could be classified, Machine Vision (MV), floor sensors and wearable sensors. MV is preferred because it is effective in continuous authentication and is the most non-intrusive approach.
	
	We will create a system for human gait recognition using machine Vision and Convolutional Neural Networks, that accept a series of frames with the person walking.
	
	\newpage
	
	\section{Contributions}
	\addcontentsline{toc}{section}{Contributions}
	\vspace{1cm}
	
	\newpage
	
	\section{State of the Art}
	\addcontentsline{toc}{section}{State of the Art}
	\vspace{1cm}
	
	Human gait is the movement pattern of the limbs during walking. It can vary depending on the persons age, weight, how tired he is and if he is carrying extra weight. A system for recognizing persons by their walking should take all of the situations from above, to correctly identify them.
	
	There are three main approaches for identifying people by their gait, Machine Vision (MV), floor sensors and wearable sensors. Each of the three approaches have some disadvantages and advantages:
		\begin{itemize}
			\item MV:
				\begin{itemize}
					\item it is cheaper to implement, no need to install extra sensors, just some video cameras;
					\item can cover a wide area;
					\item it is affected if the people are wearing voluminous clothes;
				\end{itemize}
			\item Floor Sensors:
				\begin{itemize}
					\item are not affected by the clothes worn by the user;
					\item are more expensive to implement than MV;
					\item limited area for recognizing people;
				\end{itemize}
			\item Wearable Sensors:
				\begin{itemize}
					\item are not limited by a specific area;
					\item are not affected by the clothes worn by the user;
					\item you need to have physical access or to have their cooperation.
				\end{itemize}
		\end{itemize}
	
	In Machine Vision there are tow main approaches, model-free and model-based, where the first approach uses direct image sequences, whereas the latter needs more processing of the input sequence.
	
	Molhema Mohualdeen and Magdi Baker \cite{gait-silhouette-nn} have proposed a model-based approach for the Gait Recognition problem using Region os Interest (ROI), Discrete Wavelet Transform (DWT), Edges, Gait Cycle and Neural Networks.
	ROI was used in the preprocessing phase to reduce data and extract the exact silhouette from each frame, by cropping. Next, in the feature extraction phase, they used DWT for multi-scale analysis, using diagonal, horizontal and vertical details of the three levels low pass and high pass filters on two dimensions DWT. Beside 3L-2D-DWT they used Edge Detection for magnitude and orientation and box technique for step and cycle length, using the with of the bounding box. Estimating the Gait Cycle was done by combining the silhouettes between the tow main phases of Gait and combining them together for each person and measuring the combination area and the width of the white shape boundary represents the step length. Classification was done using a Back Propagation Neural Network (BPNN).
	
	\newpage
	
	\section{Approach}
	\addcontentsline{toc}{section}{Approach}
	\vspace{1cm}
	
	\newpage
	
	\section{Conclusions}
	\addcontentsline{toc}{section}{Conclusions}
	\vspace{1cm}
	
	\newpage
	
	\section{Bibliography}
	\addcontentsline{toc}{section}{Bibliography}
	\bibliography{bibliografie}
	\bibliographystyle{ieeetr}
	
\end{document}